\documentclass[a4paper]{report}
\usepackage{fullpage}
\usepackage{amsmath}
\usepackage{hyperref}

% Default to "Computer Modern Sans"
\renewcommand{\familydefault}{\sfdefault}

\begin{document}

\section{Requirements}
\begin{itemize}
\item{Size} Approximately 1.5m by 1m
\item Fully self-sufficient
\end{itemize}

%
% SHINJI
%
\section{Hull and Superstructure}
\subsection{Requirements}


%
% JOHN
%
\section{Electonics}

\subsection{Requirements}

\begin{itemize}
\item 100W solar supply
\item Survive a whole day with no solar generation
\item Able to drive continiously
\item Routine telemetry updates via Satellite modem
\end{itemize}

\paragraph{Generation Capacity}
Main power is to be supplied by a ~1.5x1m solar panel (see below), this size of panel is rated to about 100W generation capacity. In Perth, there's an approximate factor of 4.3 to go from a panel's rated power output to the daily generation capacity (in W-h), however this factor is calcualted based on using an optimal panel angle. Since the panel will be (on average) flat, the factor will be reduced to about 85\% \ref{solarcalc_angle}.

This avaliable generation capacity must be sufficient to both power the boat during the day, and provide charging for during the night (and cloudy days). This requires that a minium of 50\% of the supply power must be reserved for charging (and ideally up to 70\% to account for losses and recharging after cloudy days).

$$
P_{\text{cont}} = 100 \times 4.3 \times 0.88 \times 0.70 \div 24 \approx 11W
$$

\paragraph{Power Storage}
To be able to continue to run at full capacity in the absense of solar generation for a whole day, we need storage equivalent to 36 hours of continual drain.
$$
E_{\text{storage}} = 11W \times 36\text{hrs} \approx 400\text{W-h} = 33\text{A-hr at 12V}
$$

\paragraph{Control Drain}
The control logic (ArduPilot, GPS, logger, modem) have a constant drain that must be accounted for before the driving capacity is determined. A rough figure can be calculated by taking the peak current draw of each unit and converting that to drawn power.
\begin{itemize}
\item{ArduPilot} 203mW (calcualted by summing peak current of all major chips and multiplying by 3.5V)
\item{GPS} 105mW (peak, during active tracking)
\item{Data Logger} 21mW (during write)
\item{Satellite Modem} 75mW average (450mA peak, but only for approximately 2 mins per hour, and supply is 5V)
\end{itemize}
The above equals a total draw of 404mW

\paragraph{Driving Drain}
The design calls for two motors (avoids complexities of a rudder), so after subtracting the control draw, the avaliable power must be split between both motors.
$$
P_{\text{motor}} = \left(11W - 0.5W\right) / 2 = 5.25W
$$

\subsection{Existing Parts}

\paragraph{Controller: ArduPilot2.8 Controller Board}

\paragraph{Modem: RockBlock+ Satellite Modem}

\subsection{New Parts}

\paragraph{Battery}
Any 40A-hr 12V lead-acid battery will work, but using two batteries may be desirable to ensure a sane weight distribution.

\begin{itemize}
\item{2x 12V 18Ah SLA Battery} 
Total capacity 36Ah, distributed across both hulls
\url{https://www.jaycar.com.au/12v-18ah-sla-battery/p/SB2490}
Price: \$80AUD (plus shipping)

\item{1x Centry NS40ZLS}
Single unit, 40Ah capacity. Could be stored in the keel 
\url{http://www.batteriesdirect.com.au/shop/product/4424/ns40zlsmf.html}
Price: \$95AUD (TODO: Shipping)
\end{itemize}

\paragraph{Solar Panel}
100W/12V solar panel, 1580mm by 808mm size. 

\url{https://pethomeoutdoor.com.au/products/12v-200w-solar-panel-kit-home-generator-caravan-camping-power-mono-charging-pwm}
Price: \$212AU (TODO: Shipping)

\paragraph{Low-Voltage Supply}
\emph{Simple DC-DC converter to turn 12V battery/panel supply into ~3.5V for boards to regulate down to 3.3V.}

\paragraph{GPS}
\emph{Any TTL-based gps would work}
"Adafruit Ultimate GPS Breakout"
\url{https://core-electronics.com.au/adafruit-ultimate-gps-breakout-66-channel-w-10-hz-updates-version-3.html}
Price: \$70AU (incl. shipping)

\paragraph{Data Logger}
"SparkFun OpenLog"
\url{https://www.sparkfun.com/products/13712}
Price: \$52.75 (incl. shipping)

\paragraph{Propulsion}
Still undecided, several options. We have a capacity of approximately 10W across both motors, this means that each will run at ~5W constant, spiking up to 10W each (when at max turn)
\begin{itemize}
\item{Blue Robotics - T100 thruster}
 Rated to 130W max draw, but max efficiency is at ~6W draw.
 Price: \$144USD (plus shipping of 35USD) per motor = \$431AUD
\item{Custom build - Motor in hull}
 Complex due to sealing requirements, but very easy to tailor to our needs (and minimum impact on hydrodynamics)
 \begin{itemize}
 \item{Prop} 50mm diameter, 3 blade prop - \url{https://hobbyking.com/en_us/boats/propellers/3-blade.html} Price: AU\$42 ea.
 \item{Motor} ??
 \item{Motor Controller} 10A 5-25V Dual Channel DC Motor Driver - \url{http://www.robotshop.com/en/10a-5-30v-dual-channel-dc-motor-driver.html} Price: AU\$30 (TODO: Shipping)
 \end{itemize}
\item{Custom build - Motor outside hull}
 Simpler than in-hull, but requires making a streamlined enclosure/mounting
 \begin{itemize}
 \item{Prop} 50mm diameter, 3 blade prop - \url{https://hobbyking.com/en_us/boats/propellers/3-blade.html} Price: AU\$42 ea.
 \item{Motor} ??
 \item{Motor Controller} 10A 5-25V Dual Channel DC Motor Driver - \url{http://www.robotshop.com/en/10a-5-30v-dual-channel-dc-motor-driver.html} Price: AU\$30 (TODO: Shipping)
 \end{itemize}
\item{Low-power water pumps}
 Relatively simple to set up, but ability to moderate output is unknown. Can place in-hull with two hoses (complex, but smooth), or external (more drag, simpler build).
\end{itemize}

\section{Firmware}
Firmware will be the \texttt{APMrover2} configuration of the open-source ArduPilot suite. However, to support telemetry updates via the Satellite modem, there will need to changes to the stock firmware.

\begin{thebibliography}{9}
\bibitem{solarcalc_angle}
 Optimum angle for solar panels in Australia
 \url{https://solarcalculator.com.au/solar-panel-angle/}
 Fetched 01-Apr-2017
\end{thebibliography}

\end{document}
